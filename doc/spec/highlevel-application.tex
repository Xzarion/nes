\subsection{Abstracts}

Our application is the well known game called \textit{Snake}, implemented
in a distributed system consisting of four nodes, based on a simple field bus protocol.
\subsubsection{Protocol}
The protocol consist of one master and multiple slaves. The master sends synchronization bytes 
and frame ID. If there is no answer after $\text{timeout} =\frac{\text{bits per byte}}
{\text{baudrate}}+\text{delay}+\text{jitter}+\text{WCET} \text{ in ms}$ the next
frame request is initiated. Moreover the master has to wait $\text{timeout}*k$ where
$k$ is factor to guarantee that also all slaves have recognized the timeout.\\
All nodes that are interested in a particular frame
read the bus, wait for the sync. byte and the frame ID to start capturing the data.
Each node may has assigned frame IDs which indicate that it will send data in the 
given slot. Before reseting to the wait state at least on timout period must pass.\\
The frame ID is followed by frame length, data and a checksum. \\
In case of a timeout slaves wait for the next frame ( sync. byte and frame ID).
In case of a checksum or parity error the message gets delivered with an failure
indication. If the readback data differs from the senddata the node stops transmitting 
and waits for the next sync. byte and frame ID.

\subsubsection{Basic Application}
In the game the user has to navigate a snake to a treat without colliding
with the snake itself or with the border of the field. Whenever the snake
reaches such a treat it grows by the size of the treat. Furthermore a new
treat is generated on a different position, but may not be placed on the
snake's body. The snake moves with a constant speed, the user can only change
the direction, immediate turnarounds are prohibited. At each treat the snake 
collects the speed will be increased by a factor of $1.1$ and the user gains 
one point. The game ends whenever the snake either collides with itself or
with the border of the field. 

\subsubsection{Role distribution}
The previously described application will be implemented on a target platform
consisting of four nodes. The communication between these nodes will be handled
by a fieldbus protocol controlled by one single busmaster.
\begin{description}
\item[NODE 0:]
Implements the busmaster and we use the two push buttons to control the up-down
movement of our snake.
\item[NODE 1:]
Node 1 indicates the game state, consisting of running (lamp deactivated) or game over
(lamp activated). Furthermore a optional goal of our application is to use the 
temperature sensors, which can be influenced by the fan, to manipulated the speed
of the snake. For example the cooler the temperature readings the slower the snake, until
it freezes to death. 
\item[NODE 2:]
Implements the gamelogic and the graphical display on the TFT. The snake consists 
simply of a bunch of boxes chained together to form a snake with a size of NxM 
pixels each. At each gamestep this node paints a new box at the new head position
of the snake and deletes the box at the old tail position. When the snake reaches
a treat, the tail position will not be deleted and a new treat with the size of 
one NxM box will be drawn on a random position. The duration between two snake 
movements will be 500ms. The duration will be decreased to 454ms after the first
acheived point and so on. The following subsection describes the required system
messages.
\item[NODE 3:]
Node 3 contains a LED matrix, which will be used to display the current score. 
Furthermore it offers two additional pushbuttons which will be used to control
the left- and right direction of the snake.
\end{description}

 

\subsubsection{System messages}

The following state-messages will be sent in the system:

\begin{itemize}
  \item \textbf{Node 0 sends BTN\_UP\_DOWN\_PRESSED to Node 2}
  \item \textbf{Node 2 sends FREEZE\_LEVEL to Node 1}
  \item \textbf{Node 2 sends DEAD to Node 1}
  \item \textbf{Node 3 sends BTN\_LEFT\_RIGHT\_PRESSED to Node 2}
  \item \textbf{Node 2 sends SCORE to Node 3}
\end{itemize}

Since all described messages are state messages, each of them must be sent and
delivered between two consecutive game steps. The duration between such two 
steps in the worst case decreases by factor $1/1.1$, so the end-to-end delay
of one communication round must be bounded by 50ms to guarantee a fluent gameflow. 

